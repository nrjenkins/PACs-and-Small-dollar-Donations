% Nicholas R. Jenkins
% Department of Political Science
% University of California, Riverside
%
%  Resaerch Proposal
%  ------------------------------------------------------------------------------

% Setup Document and Formatting %
\documentclass[12pt]{article}

% Load Packages %
\usepackage[utf8]{inputenc}
\usepackage[english]{babel}
\usepackage{hyperref}
\usepackage[dvipsnames]{xcolor}
\usepackage[doublespacing]{setspace} % For double spacing
%\usepackage[singlespacing]{setspace} % For single spacing
\usepackage[margin = 1in]{geometry}
\usepackage{bookmark}
\usepackage{ntheorem}
\usepackage{amsmath}
\usepackage{bm}
\usepackage{tikz}
\usepackage{longtable}
\usepackage[titletoc,title]{appendix}
\usepackage[section]{placeins}
\usepackage[bottom]{footmisc}
\usetikzlibrary{shapes, shadows, arrows}



% Reference Setup %
\hypersetup{
    pdftitle={},    			% title
    pdfauthor={Nicholas R. Jenkins},     		% author
    pdfsubject={Research Proposal},   		% subject of the document
    pdfkeywords={}, 	% list of keywords
    pdfnewwindow=true,      		% links in new window
    colorlinks=false,       			% false: boxed links; true: colored links
    linkcolor=blue,          			% color of internal links
    citecolor=blue,        			% color of links to bibliography
    filecolor=blue,      				% color of file links
    urlcolor=blue           			% color of external links
}

\definecolor{light-gray}{gray}{0.75}


% Bibliography %
\usepackage[authordate, natbib, isbn=false, url=false, doi=false, backend=biber]{biblatex-chicago} 
\bibliography{/Users/nick/Documents/Research/References/BibTeX/biblatex.bib}

% Clear Data Accessed Line in Biblography
\AtEveryBibitem{%
  \ifentrytype{online}
    {}
    {\clearfield{urlyear}\clearfield{urlmonth}\clearfield{urlday}}}
    
\DeclareCiteCommand{\citeauthorfirstlast}
  {\boolfalse{citetracker}%
   \boolfalse{pagetracker}%
   \DeclareNameAlias{labelname}{first-last}%
   \usebibmacro{prenote}}
  {\ifciteindex
     {\indexnames{labelname}}
     {}%
   \printnames{labelname}}
  {\multicitedelim}
  {\usebibmacro{postnote}}


% Hypotheses %
\newtheorem{hyp}{Hypothesis}

\makeatletter
\newcounter{subhyp} 
\let\savedc@hyp\c@hyp
\newenvironment{subhyp}
 {%
  \setcounter{subhyp}{0}%
  \stepcounter{hyp}%
  \edef\saved@hyp{\thehyp}% Save the current value of hyp
  \let\c@hyp\c@subhyp     % Now hyp is subhyp
  \renewcommand{\thehyp}{\saved@hyp\alph{hyp}}%
 }
 {}
\newcommand{\normhyp}{%
  \let\c@hyp\savedc@hyp % revert to the old one
  \renewcommand\thehyp{\arabic{hyp}}%
} 
\makeatother


% Title Page Setup %
\title{\textbf{Rejecting PACs and Expecting Small-dollars: An Examination of the Effects of Rejecting PAC Donations}}
\author{Nicholas R. Jenkins\thanks{Graduate Student, Department of Political Science, University of California, Riverside, Riverside CA 92521. Email: \href{mailto:nicholas.jenkins@email.ucr.edu}{nicholas.jenkins@email.ucr.edu}}}
\date{\today}


% Document %
\begin{document}

% Title Page %
\maketitle
\thispagestyle{empty}

\pagebreak

\cleardoublepage
\setcounter{page}{1}

% Instructions
% For a topic of your choosing, you should do additional reading to generate a research design. The design should be roughly 12 pages (double spaced, Times New Roman 12 point font) and contain a clear statement of the problem being addressed, a critical review of the literature, and a research plan that discusses how you propose to advance knowledge on this subject. Essentially, this is the front end of a paper up to the results section. This includes a complete discussion of where your data will come from and how they will be analyzed. For example, if your design involves an experiment, you must provide a complete description of sample selection and criteria, all treatment materials, etc. Similarly, if you plan to use existing survey data, you must detail how those data will be analyzed and any additional data you might collect on your own (e.g., contextual variables, etc.). Your design may involve data collection that you currently cannot do (e.g., you lack funds to conduct a field experiment or a survey), but it must be feasible (e.g., you cannot propose an experiment in which you physically manipulate gender).


% Section 1: Introduction  %
\section{Introduction} \label{sec: intro}
% Introduction: A clear, concise statement of the puzzle you are addressing, of your proposed resolution, and of the empirical work you will do.

On national television during the Democratic debate on February 6, 2016, Bernie Sanders stood on stage and enthusiastically announced to the audience, ``I am very proud to be the only candidate up here who does not have a super PAC, who’s not raising huge sums of money from Wall Street and special interests." Indeed, during the campaign, Sanders publicly voiced his opposition to PAC contributions and refused to accept their donations while his Democratic opponent, Hillary Clinton, accepted over \$1.7 million from political action committees (PACs) for which she faced sharp criticism from voters on the left \citep{harper_2020_2019, ye_hee_lee_sanderss_2016, seitz-wald_promise_2015, bump_why_2016}. 

This trend in rejecting PAC contributions has continued into the 2018 Congressional races and so far in the 2020 presidential races. For example, \citet{evers-hillstrom_democrats_2018} reported in OpenSecrets News that fifty-two members of the 116 Congress, including thirty-five non-incumbents, announced that they would not accept money from PACs. Similarly, \citeauthorfirstlast{harper_2020_2019} wrote an article in ABC News claiming, ``The 2020 Democratic presidential candidates are forgoing corporate money in an effort to capture small donors.'' In fact, as of April of 2019, all fourteen of the 2020 presidential Democratic candidates have declined to accept corporate PAC contributions (although only three have declined to accept contributions from all PACs). 

This push for campaign donations in the form small-dollar donations, rather than PAC contributions, is indicative of a demand from voters to bring the era of ``captured" politicians to an end. Indeed, the ideal candidate for most voters would likely be one that manages to raise money from a large number of individual donors and rejects donations from special interests and PAC to prevent themselves from becoming beholden to these donors rather than their constituents. In fact, \citet{bowler_campaign_2016} show that the sources of such funding significantly influence attitudes towards campaign money.  

Despite this shift away from PAC donations, however, is Harper's \citeyear{harper_2020_2019} claim that these strategies represent an attempt to galvanize individuals to donate true? That is, does publicly rejecting PAC contributions mobilize voters to donate to political campaigns?  In this article, I argue that announcing opposition to PAC donations serves as a signal to voters that a candidate is more trustworthy and more interested in being a representative of the people,  which will in-turn lead to more donations from individual voters. 

To test this claim, I employ a survey experiment that confronts respondents with two candidate statements. Each candidate's statement list some similar and generic demographic information, policy positions, and goals. The critical difference is that one candidate's statement explains that he does not accept donations from PACs and that small-dollar individual donations have primarily funded his campaign. The other statement makes no mention of campaign finance. 

After conducting a difference of proportions test, I expect to find that respondents will be more willing to donate to the candidate who does not accept PAC money and for this candidate to be rated as more trustworthy, honest, and a better representative of the people. 

This study will make two significant contributions. First, it will advance our understanding of how perceptions about candidates motivate political action. Specifically, how perceptions about campaign money motivate political action. To date, very little is known about how campaign financing shapes attitudes towards Congress and the presidency (see \cite{dowling_super_2014, bowler_campaign_2016}). Hence, it will advance our understanding in this area by testing statements made by candidates about campaign finance sources in presidential elections.

Second, this study is the first to test the implicit causal claim that voters want candidates who are beholden to the people and not ``bought-out" by special interests and corporate lobbyists. The growing trend of candidates refusing to accept campaign contributions from PACs assumes this fact to be true, and that by publicizing their opposition to PACs they will attract more political support and small-dollar donations from individuals.  


\section{Perceptions about Money in Politics}

 Americans are remarkably cynical about Congress, and this cynicism is increasing. According to a survey conducted by the Pew Research Center, as of February of 2014 only twenty-four percent of the public trust Congress \citep{pew_research_center_public_2014}. The decline of trust in Congress has been trending downward since 1950 \citep{dalton_social_2005}. Among the sources of distrust, researchers have argued that perceptions of corruption and money in politics play a crucial role \citep{persily_perceptions_2004, vanheerde-hudson_parties_2013}. 
 
 Although a lot is known about voting in congressional \citep{abramowitz_explaining_1988, welch_effects_1997} and presidential races \citep{petrocik_issue_1996, nadeau_national_2001, polsby_landmarks_2002}, less is known about the effects of campaign money on attitudes and support for presidential candidates. Specifically, do the sources of funding that a candidate receives affect their public support? 
 
 Previous research has investigated the role of money in politics and campaigns, which has given us a better understanding of who donates to campaigns. For example, scholars have examined the role of campaign finance legislation and the FEC and how they limit or fail to limit, undue influence of large donors  \citep{magleby_money_2010, raja_small_2008}. Researchers have also investigated why donors donate, to what extent their donations influence policy, and what their donation strategies are \citep{francia_financiers_2003}. Most relevant to this study, \citet{alexander_good_2005} correlated campaign funding sources with candidate vote share. He finds that PAC and in-state donations are correlated with a candidate's vote share.  Unfortunately, he did not consider individual donations, and his study covered Congressional elections, not presidential, from 1998 to 2002. Since then, the amount of money in politics has increased, and candidates have begun rejecting donations from PACs. 
 
 More recently, researchers have begun investigating how an individual's perception about money in politics. For example, \citet{bowler_campaign_2016} conduct a survey experiment and find that beliefs about donations are dependent on partisanship and information about the source of the donations. In the end, they conclude that attitudes about campaign contributions are informed by much more than just perceptions of corruption. This suggests that the source of the contributions does matter to voters and that messages about campaign finance may alter their perception of candidates. Indeed, by announcing opposition to PAC donations, candidates provide an information signal to voters that PACs donations produce adverse effects and that pursuing finance through alternative means is a more ``ethical" strategy. 
 
 That American's lack political knowledge is perhaps the most well established finding in political science \citep{page_rational_1992, carpini_what_1997}. Hence, there is little reason to suspect that voters are knowledgeable about a particular candidate's campaign contribution sources unless they are explicitly discussed. However, voters typically think that contributions to candidates from corporations are more corrupt than contributions from individuals \citep{bowler_campaign_2016}. Moreover, the majority of Americans are dissatisfied with campaign practices in general, \citep{mayer_public_2001, persily_perceptions_2004}. Since trust in political officials has been decline over the last several decades, candidates should be able to use this to their advantage on the campaign trail.
 
 Given that American's are typically unaware of political information and that they tend to be distrusting of campaign financing, candidates can strategically advertise the fact that they will not accept PAC contributions. By doing so, they can prime voters to interpret this message as saying that rejecting PAC money means a more trustworthy and honest candidate. Such priming effects through the media have been well-documented \citep{iyengar_news_1989}. Indeed, Ella \citet{nilsen_race_2019} reported that Elizabeth Warren ``swore off PAC money to make a statement" in a story for Vox.com. In an email to her supporters Warren explained, ``For every time you see a presidential candidate talking with voters at a town hall, rally, or local diner, those same candidates are spending three or four or five times as long with wealthy donors — on the phone, or in conference rooms at hedge fund offices, or at fancy receptions and intimate dinners — all behind closed doors" \citep{nilsen_race_2019}. The 2020 Democratic candidates are essentially in competition to distance themselves as far as they can from PAC and lobbyist donations. 
 
 Campaigns are still extraordinarily expensive, however, and rejecting contributions from large donors means that the money needs to come from somewhere else. Further, Democratic candidates like Bernie Sanders are soliciting donations from individuals for small amounts by advertising the fact that the average donation made to the campaign in 2020 is twenty-seven dollars \citep{gambino_not_2019}. This strategy is not only meant to galvanize supporters to donate but since voters see small-dollar donations from individuals as more honest \citep{bowler_campaign_2016} it also creates an image of honesty and trustworthiness. These strategies are risky and require candidates to make-up for the lack of PAC contributions with these small-dollar donations from individuals. Ella \citet{nilsen_race_2019} called Warren's more extreme strategy of saying no to, ``... PAC[s], corporate PAC[s], or Super PAC money, [and] no donations from federal lobbyists" a ``risky move." These candidates must be hoping that their public stances against large-dollar donations will excite their supporters to enough to get them to donate to their campaigns.
 
 Do these appeals to individual supporters for donations actually motivate them to action? Does publicly announcing that one will reject PAC donations rally support and increase small-dollar donations? With these questions, and the considerations above, in mind, I formulate the following hypotheses:
 
 \begin{hyp}
    Individuals will be more likely to donate to candidates that publicize that they will not accept campaign contributions from PACs.
\end{hyp}

\begin{hyp}
    Candidates that publicize that they will not accept campaign contributions from PACs will be perceived as more trustworthy.
\end{hyp}

\begin{hyp}
    Candidates that publicize that they will not accept campaign contributions from PACs will be perceived as more honest.
\end{hyp}

\begin{hyp}
    Candidates that publicize that they will not accept campaign contributions from PACs will be perceived as better representatives of the people.
\end{hyp}


\section{Data and Method}

\subsection{Data Collection}

In order to test the effects of PAC opposition on small-dollar donations, I will use an online survey experiment administered with Mechanical Turk (MTurk). Amazon's MTurk platform is an effective way to field survey experiments and has been shown to produce samples that are more representative of the US population that in-person convenience samples \citep{berinsky_using_2011}. 

After the initial recruitment, respondents will be faced with several demographic questions about their age, race, ethnicity, sex, and party identification. Following these questions, they will see the bios for two fictitious candidates for public office. These bios are shown in Figures \ref{fig: pac candidate} and \ref{fig: control candidate}. Figure \ref{fig: pac candidate} shows that candidate that expresses general policy goals and positions along with some basic background information. However, this candidate also includes a short description expressing their opposition to campaign contributions from PACs and emphasizes the fact that the majority of their financial support has come from small-dollar donations made by individuals.  

\begin{figure}[ht]
    \centering
    \includegraphics[angle=270, width=1\textwidth]{../../Data/Candidate_1.pdf}
    \caption{\textbf{Candidate That Publicizes Their Opposition to PAC Donations.}}
    \label{fig: pac candidate}
\end{figure} 

The candidate in Figure \ref{fig: control candidate} also outlines a basic description of some policy goals along with some background information that is very similar to the candidate in Figure \ref{fig: pac candidate}. The key difference here is that this second candidate does not say anything about campaign finance or their position on accepting money from PACs. Since this is the only significant difference, it should allow me to effectively isolate the treatment effect of publicizing opposition to PAC donations. 

\begin{figure}[ht]
    \centering
    \includegraphics[angle=270, width=1\textwidth]{../../Data/Candidate_2.pdf}
    \caption{\textbf{Candidate That Does Not Publicize Their Opposition to PAC Donations.}}
    \label{fig: control candidate}
\end{figure} 

These two bios will be presented to side-by-side to each respondent. After reading them, respondents will be asked the following questions:

\begin{enumerate}
    \item How often do you watch or read the news?
    \item Which candidate would you be more likely to make a donation to their campaign? 
    \item Which candidate do you think is more trustworthy?
    \item Which candidate is more honest?
    \item Which candidate would be a better representative of the people?
    \item From 1 to 5, with 1 being the highest, how interested would you say that you are in politics?
\end{enumerate} 

These questions will allow me to identify what the effects of publicizing opposition to PAC donations are with respect to motivating individuals to contribute to a campaign, as well as their perceptions of trustworthiness and honesty. I expect to find that respondents will be more willing to donate to the candidate that publicizes their opposition and that they will also be rated as more trustworthy and honest. 

This survey will be administered through MTurk, and I will aim to recruit at least three hundred participants. I will plan to run the survey for as long as needed to achieve an appropriate sample size. In sum, this survey will collect data about each respondent on the following variables: age, race, ethnicity, sex, party identification, level of political interest, level of political engagement, which candidate they are more likely to donate to, which candidate they perceive as being more trustworthy, which candidate they perceive as being more honest, and which candidate they perceive as being a better representative of the people.

\subsection{Dependent Variables}

The dependent variables will be the binary responses to the questions about which candidate the respondent would be more likely to donate to, which one they believe to be more trustworthy, which one they believe to be more honest, and which one they believe would be a better representative of the people. Each of these variables will be analyzed separately.  

\subsection{Methodology}

To test these hypotheses, I will calculate the proportion of responses for each candidate and then conduct a difference of proportions z-test. Since I will have a random sample of respondents, and they only meaningful difference between each candidate's statement is the fact that one states opposition to PACs and the other does not, this simple test should allow me to identify a causal effect accurately. 


\section{Results}

In this section, I will present the findings from the survey experiment. I will provide a table for each dependent variable and discuss them individually.


\section{Conclusion}

Refusing to accept contributions from PACs is a growing trend, especially among Democrats. Indeed, as of April 2015, all fourteen of the 2020 Democratic candidates have refused to accept corporate PAC contributions. Considering that trust in elected officials among Americans is on a continual decline and that the individuals perceive large donations from PACs as more corrupt, the strategy is likely an attempt to signal to voters that their candidacy cannot be bought and that they are stronger representative of the people. Thus, it is expected that when a candidate publicizes the fact that they refuse to accept contributions from PACs, voters will perceive them to be more trustworthy, honest, and a better representative of the people. 

Since campaigns still require an extraordinary amount of money, this strategy also involves a considerable amount of risk. Candidates will need to find alternative ways to raise the funds they need. Indeed, candidates have connected the narrative of not being ``bought" by PAC donations with the narrative that small-dollar donations from individual donors result in a more honest campaign. They have made these appeals by advertising the percentage of their campaign's donations that come from individuals and by bragging about have single-digit dollar amounts as their average donation. Are these appeals successful? Does publicly rejecting PAC contributions convey the message that a candidate is trustworthy and increase the likelihood of attracting donations from individual donors? 

In this article, I propose a survey experiment to test these claims. Using MTurk to recruit participants, I will ask them a series of questions about two hypothetical candidates. Both candidates share similar backgrounds, but one candidate explicitly mentions that he will not accept donations from PACs. After reading through these candidate statements, respondents are asked which candidate that they would be more likely to donate money, along with questions about which candidate they believe to be more trustworthy, honest, and representative of the people, rather than special interests. 

This is the first study to investigate the political effects of publicly opposing PAC donations. Although much has been written about campaign contributions, little is known about how the sources of campaign donations motivate political action, especially for presidential elections. Additionally, this study is the first to test the implicit claims increasing made by candidates that voters will be more likely to donate to candidates that they believe to be honestly trying to represent the will of the majority and resist undue financial influence. 



%\section{Implications}
% Implications: Explain what you expect the completed project will add to our understanding of some broader set of analytical or empirical issues in IPE.




\pagebreak
\pdfbookmark[1]{References}{References}
%\bibliography{/Users/nick/Documents/Research/Articles/reference.bib}
\printbibliography
\pagebreak


\begin{appendices}



\end{appendices}


\end{document}
