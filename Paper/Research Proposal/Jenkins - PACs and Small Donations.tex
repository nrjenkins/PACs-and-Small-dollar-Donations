% Nicholas R. Jenkins
% Department of Political Science
% University of California, Riverside
%
%  Research Proposal
%  ------------------------------------------------------------------------------

% Setup Document and Formatting %
\documentclass[12pt]{article}

% Load Packages %
\usepackage[utf8]{inputenc}
\usepackage[english]{babel}
\usepackage{hyperref}
\usepackage[dvipsnames]{xcolor}
%\usepackage[doublespacing]{setspace} % For double spacing
\usepackage[singlespacing]{setspace} % For single spacing
\usepackage[margin = 1in]{geometry}
\usepackage{bookmark}
\usepackage{ntheorem}
\usepackage{amsmath}
\usepackage{bm}
\usepackage{tikz}
\usepackage{longtable}
\usepackage[titletoc,title]{appendix}
\usepackage[section]{placeins}
\usepackage[bottom]{footmisc}
\usetikzlibrary{shapes, shadows, arrows}



% Reference Setup %
\hypersetup{
    pdftitle={},    			% title
    pdfauthor={Nicholas R. Jenkins},     		% author
    pdfsubject={Research Proposal},   		% subject of the document
    pdfkeywords={}, 	% list of keywords
    pdfnewwindow=true,      		% links in new window
    colorlinks=false,       			% false: boxed links; true: colored links
    linkcolor=blue,          			% color of internal links
    citecolor=blue,        			% color of links to bibliography
    filecolor=blue,      				% color of file links
    urlcolor=blue           			% color of external links
}

\definecolor{light-gray}{gray}{0.75}


% Bibliography %
\usepackage[authordate, natbib, isbn=false, url=false, doi=false, backend=biber]{biblatex-chicago} 
\bibliography{/Users/nick/Documents/Research/References/BibTeX/biblatex.bib}

% Clear Data Accessed Line in Biblography
\AtEveryBibitem{%
  \ifentrytype{online}
    {}
    {\clearfield{urlyear}\clearfield{urlmonth}\clearfield{urlday}}}
    
\DeclareCiteCommand{\citeauthorfirstlast}
  {\boolfalse{citetracker}%
   \boolfalse{pagetracker}%
   \DeclareNameAlias{labelname}{first-last}%
   \usebibmacro{prenote}}
  {\ifciteindex
     {\indexnames{labelname}}
     {}%
   \printnames{labelname}}
  {\multicitedelim}
  {\usebibmacro{postnote}}


% Hypotheses %
\newtheorem{hyp}{Hypothesis}

\makeatletter
\newcounter{subhyp} 
\let\savedc@hyp\c@hyp
\newenvironment{subhyp}
 {%
  \setcounter{subhyp}{0}%
  \stepcounter{hyp}%
  \edef\saved@hyp{\thehyp}% Save the current value of hyp
  \let\c@hyp\c@subhyp     % Now hyp is subhyp
  \renewcommand{\thehyp}{\saved@hyp\alph{hyp}}%
 }
 {}
\newcommand{\normhyp}{%
  \let\c@hyp\savedc@hyp % revert to the old one
  \renewcommand\thehyp{\arabic{hyp}}%
} 
\makeatother


% Title Page Setup %
\title{\textbf{Rejecting PACs and Expecting Small-dollars: An Examination of the Effects of Rejecting PAC Donations}}
\author{Nicholas R. Jenkins\thanks{Graduate Student, Department of Political Science, University of California, Riverside, Riverside CA 92521. Email: \href{mailto:nicholas.jenkins@email.ucr.edu}{nicholas.jenkins@email.ucr.edu}}}
\date{\today}


% Document %
\begin{document}

% Title Page %
\maketitle
\thispagestyle{empty}

\pagebreak

\cleardoublepage
\setcounter{page}{1}

% Instructions
% For a topic of your choosing, you should do additional reading to generate a research design. The design should be roughly 12 pages (double spaced, Times New Roman 12 point font) and contain a clear statement of the problem being addressed, a critical review of the literature, and a research plan that discusses how you propose to advance knowledge on this subject. Essentially, this is the front end of a paper up to the results section. This includes a complete discussion of where your data will come from and how they will be analyzed. For example, if your design involves an experiment, you must provide a complete description of sample selection and criteria, all treatment materials, etc. Similarly, if you plan to use existing survey data, you must detail how those data will be analyzed and any additional data you might collect on your own (e.g., contextual variables, etc.). Your design may involve data collection that you currently cannot do (e.g., you lack funds to conduct a field experiment or a survey), but it must be feasible (e.g., you cannot propose an experiment in which you physically manipulate gender).


% Section 1: Introduction  %
\section{Introduction} \label{sec: intro}
% Introduction: A clear, concise statement of the puzzle you are addressing, of your proposed resolution, and of the empirical work you will do.


On national television during the Democratic debate on February 6, 2016, Bernie Sanders stood on stage and enthusiastically announced to the audience, ``I am very proud to be the only candidate up here who does not have a super PAC, who’s not raising huge sums of money from Wall Street and special interests." Indeed, during the campaign, Sanders publicly voiced his opposition to PAC contributions and refused to accept their donations while his Democratic opponent, Hillary Clinton, accepted over \$1.7 million from political action committees (PACs) for which she faced sharp criticism from voters on the left \citep{harper_2020_2019, ye_hee_lee_sanderss_2016, seitz-wald_promise_2015, bump_why_2016}. 

This trend in rejecting PAC contributions has continued into the 2018 Congressional races and so far in the 2020 presidential races. For example, \citet{evers-hillstrom_democrats_2018} reported in OpenSecrets News that fifty-two members of the 116 Congress, including thirty-five non-incumbents, announced that they would not accept money from PACs. Similarly, \citeauthorfirstlast{harper_2020_2019} wrote an article in ABC News claiming, ``The 2020 Democratic presidential candidates are forgoing corporate money in an effort to capture small donors.'' In fact, as of April of 2019, all fourteen of the 2020 presidential Democratic candidates have declined to accept corporate PAC contributions (although only three have declined to accept contributions from all PACs). 

The push for campaign donations in the form of small-dollar donations, rather than PAC contributions, is indicative of a demand from voters to bring the era of ``captured" politicians to an end. Indeed, the ideal candidate for most voters would likely be one that manages to raise money from a large number of individual donors and rejects donations from special interests and PAC to prevent themselves from becoming beholden to these donors rather than their constituents. In fact, \citet{bowler_campaign_2016} show that the sources of such funding significantly influence attitudes towards campaign money.  

Despite this shift away from PAC donations, however, is Harper's \citeyear{harper_2020_2019} claim that these strategies represent an attempt to galvanize individuals to donate true? That is, does publicly rejecting PAC contributions mobilize voters to donate to political campaigns? Additionally, how does the rejection of PACs affect voter's perceptions of candidates?  In this article, I examine these questions and argue that rejecting PAC donations will encourage more individual contributions and that it sends a signal to voters that a candidate is more trustworthy.

To test these claims, I first use Federal Election Commission data on campaign donation data for all candidates in the 2018 Congressional House Elections to examine the effect of announcing opposition to PAC donations on individual contributions. Second, I employ a survey experiment that confronts respondents with two candidate statements. Each candidate's statement list some similar and generic demographic information, policy positions, and goals. The critical difference is that one candidate's statement explains that he does not accept donations from PACs and that small-dollar individual donations have primarily funded his campaign. The other statement makes no mention of campaign finance. Following these vignettes, respondents will be asked which candidate they think is more trustworthy, which would be a better representative of the people, and to which candidate they would be most likely to make a campaign donation. 

This study will make two significant contributions. First, it will advance our understanding of how perceptions about candidates motivate political action. Specifically, how perceptions about campaign money motivate political action. To date, very little is known about how campaign financing shapes attitudes towards Congress and the presidency (see \cite{dowling_super_2014, bowler_campaign_2016}). Hence, it will advance our understanding in this area by testing statements made by candidates about campaign finance sources in presidential elections.

Second, this study is the first to test the implicit causal claim that voters want candidates who are beholden to the people and not ``bought-out" by special interests and corporate lobbyists. The growing trend of candidates refusing to accept campaign contributions from PACs assumes this fact to be true and that by publicizing their opposition to PACs, they will attract more political support and small-dollar donations from individuals.  


\section{Perceptions about Money in Politics}

 Americans are remarkably cynical about Congress, and this cynicism is increasing. According to a survey conducted by the Pew Research Center, as of February of 2014, only twenty-four percent of the public trust Congress \citep{pew_research_center_public_2014} and public trust in Congress has been declining since 1950 \citep{dalton_social_2005}. Among the sources of distrust, researchers have argued that perceptions of corruption and money in politics play a crucial role \citep{persily_perceptions_2004, vanheerde-hudson_parties_2013}. 
 
 Although a lot is known about voting in congressional \citep{abramowitz_explaining_1988, welch_effects_1997} and presidential races \citep{petrocik_issue_1996, nadeau_national_2001, polsby_landmarks_2002}, less is known about the effects of campaign money on attitudes and support for presidential candidates. Specifically, do the sources of funding that a candidate receives affect their public support? 
 
 Previous research has investigated the role of money in politics and campaigns, which has given us a better understanding of who donates to campaigns. For example, scholars have examined the role of campaign finance legislation and the FEC and how they limit or fail to limit, undue influence of large donors  \citep{magleby_money_2010, raja_small_2008}. Researchers have also investigated why donors donate, to what extent their donations influence policy, and what their donation strategies are \citep{francia_financiers_2003}. Most relevant to this study, \citet{alexander_good_2005} correlated campaign funding sources with candidate vote share. He finds that PAC and in-state donations are correlated with a candidate's vote share. In a study of small-dollar donations in Congressional elections, \citep{culberson_small_2019} find that small-dollar contributions are distributed rather evenly across candidates, including non-incumbents. Similarly, they find that small-dollar donations are more numerous in competitive elections and among more ideologically extreme candidates. \citep{culberson_small_2019} conclude by arguing that the participation of small-dollar donors ``may help with democratization of the electoral process by expanding participation in campaign financing, with money flowing more equally to all candidate types."   
 
 More recently, researchers have begun investigating how individuals perceive money in politics. For example, \citet{bowler_campaign_2016} conduct a survey experiment and find that beliefs about donations are dependent on partisanship and information about the source of the donations. In the end, they conclude that attitudes about campaign contributions are informed by much more than just perceptions of corruption. This suggests that the source of the contributions does matter to voters and that messages about campaign finance may alter their perception of candidates. Indeed, by announcing opposition to PAC donations, candidates may provide an information signal to voters that PAC donations produce adverse effects and that pursuing financing through alternative means is a more ``ethical" strategy. 
 
 That American's lack political knowledge is perhaps the most well established finding in political science \citep{page_rational_1992, carpini_what_1997}. Hence, there is little reason to suspect that voters are knowledgeable about a particular candidate's campaign contribution sources unless they are explicitly discussed. However, voters typically think that contributions to candidates from corporations are more corrupt than contributions from individuals \citep{bowler_campaign_2016}. Moreover, the majority of Americans are dissatisfied with campaign practices in general, \citep{mayer_public_2001, persily_perceptions_2004}. Since trust in political officials has been declined over the last several decades, candidates should be able to use this to their advantage on the campaign trail.
 
 Given that Americans are typically unaware of political information and that they tend to be distrusting of campaign financing, candidates can strategically advertise the fact that they will not accept PAC contributions. By doing so, they can prime voters to interpret this message as saying that rejecting PAC money means a more trustworthy and honest candidate. Such priming effects through the media have been well-documented \citep{iyengar_news_1989}. Indeed, Ella \citet{nilsen_race_2019} reported that Elizabeth Warren ``swore off PAC money to make a statement" in a story for Vox.com. In an email to her supporters Warren explained, ``For every time you see a presidential candidate talking with voters at a town hall, rally, or local diner, those same candidates are spending three or four or five times as long with wealthy donors — on the phone, or in conference rooms at hedge fund offices, or at fancy receptions and intimate dinners — all behind closed doors" \citep{nilsen_race_2019}. The 2020 Democratic candidates are almost in competition to distance themselves as far as they can from PACs and special interests. 
 
 Campaigns are still extraordinarily expensive, however, and rejecting contributions from large donors means that the money needs to come from somewhere else. Further, Democratic candidates like Bernie Sanders are soliciting donations from individuals for small amounts by advertising the fact that the average donation made to the campaign in 2020 is twenty-seven dollars \citep{gambino_not_2019}. This strategy is not only meant to galvanize supporters to donate, but since voters see small-dollar donations from individuals as more honest \citep{bowler_campaign_2016} it also creates an image of honesty and trustworthiness. These strategies are risky and require candidates to make-up for the lack of PAC contributions with these small-dollar donations from individuals. Ella \citet{nilsen_race_2019} called Warren's more extreme strategy of saying no to, ``... PAC[s], corporate PAC[s], or Super PAC money, [and] no donations from federal lobbyists" a ``risky move." These candidates must be hoping that their public stances against large-dollar donations will excite their supporters enough to get them to donate to their campaigns.

%\section{Implications}
% Implications: Explain what you expect the completed project will add to our understanding of some broader set of analytical or empirical issues in IPE.




\pagebreak
\pdfbookmark[1]{References}{References}
%\bibliography{/Users/nick/Documents/Research/Articles/reference.bib}
\printbibliography
\pagebreak


\begin{appendices}



\end{appendices}


\end{document}
